\chapter{Introdução à Lógica}\label{cap:IntroLogic}

\epigraph{``Se você sabe que está morto, você está morto.\\ Mas se você sabe que está morto, você não está morto,\\ portanto você não sabe se está morto ou não.''}{Zenão de Citio (334-262 a.C.)}

\section{O que é Lógica?}\label{sec:WhatIsLogic}

Antes de apresentar uma descrição histórica da lógica, este texto começa pela árdua tarefa de apresentar de forma sucinta uma resposta para a pergunta, \textbf{``o que é a lógica?}''. Como dito em \cite{benja-Logica, copi1981}, a palavra lógica e suas derivações são familiares a quase todas (se não todas) as pessoas, de fato, é comum durante o cotidiano do dia a dia as pessoas recorrerem ao uso do termo lógica ou de seu derivados, sendo que na maioria das vezes seu uso está ligada à ideia de obviedade (ou certeza), por exemplo, nas frases:

\begin{itemize}
	\item[(a)] É lógico que vou na festa.
	\item[(b)] É lógico que ciência da computação é um curso difícil.
	\item[(c)] Logicamente o Vasco não pode ganhar o título da primeira divisão nacional em 2021.
	\item[(d)] Logicamente se eu tomar banho, vou ter que me molhar.
\end{itemize}

Essa forma de usar os derivados da palavra lógica enquanto entidades para transmissão de certeza pode ser usada como gatilho ``fácil e preguiçoso'' para enunciar que a lógica se trata de uma ciência (ou disciplina) acerca das certezas sobre os fatos do mundo real.

Existem outras respostas comumente encontradas na literatura acadêmica (ver \cite{abe2002-logica, benja-Logica, joaoPavao2014}) para o que seria a lógica, entre essas respostas, esta aquela que descreve a lógica como sendo um mecanismo utilizado durante o raciocínio estruturado e correto\sidefootnote{Uma visão semelhante a esta é descrita em \cite{magnus2020}, que diz que a lógica está preocupada com a avaliação de argumentos, com a separação dos argumentos bons dos argumentos mal (ou ruins).}, isto é, uma ferramenta do raciocínio que possibilita a inferência de conclusões a partir de premissas \cite{abe2002-logica, copi1981, hodges1997}, por exemplo, dado as premissas:

\begin{itemize}
	\item[(a)] Toda quinta-feira é servido peixe no almoço.
	\item[(b)] Hoje é quarta-feira.
\end{itemize}

O raciocínio munido da ``ferramenta de inferência'' contida na lógica permite deduzir a afirmação: \textbf{Amanhã será servido peixe no almoço}, como conclusão. Note que esta segunda resposta estabelece que a lógica é um tipo de procedimento mental capaz de transformar (informações) as entradas (as premissas) nas saídas (a conclusão), usando uma ferramenta de inferência.

Essas duas formas de encarar a lógica não estão totalmente erradas, entretanto, também não exibem de forma completa o real significado do que seria a lógica em si. Uma terceira resposta para a pergunta ``O que é a lógica?'' aparece na edição de 1953 da Encyclopædia Britannica na seguinte forma: ``\textit{Logic is the systematic study of the structure of propositions and of the general conditions of valid inference by a method which abstracts from the content or matter of the propositions and deals only with their logical form}''. Note que essa resposta utiliza-se de autorreferência\sidefootnote{Autorreferência é um fenômeno que ocorre na língua natural e nas linguagens formais, tal fenômeno consiste em uma oração ou fórmula que se refere a si mesma de forma direta ou através de alguma sub-frase ou fórmula intermediária, ou ainda por meio de alguma codificação.}, pois a mesma tenta definir o que é a lógica em função do termo ``forma lógica''.

Apesar dessa definição recursiva, a resposta da Encyclopædia Britannica apresenta duas características muito marcantes para a apresentação da lógica enquanto ciência (ou disciplina) nos dias atuais. A primeira característica é a validade das afirmações derivadas (ou concluídas) pelos mecanismos de inferência. A segunda característica é a importância da forma de representação (a escrita) dos termos lógicos.

A validade remonta a ideia de um significado dual (verdadeiro e falso) para as afirmações, ou seja, fornece indícios da existência de interpretações das afirmações, e isto significa que existem diferentes significados para as afirmações a depender de um fator que pode ser chamado de contexto, por exemplo, considere a seguinte afirmação:

\begin{center}
	``\textbf{O atual presidente americano é um democrata}''.
\end{center}

Note que o contexto temporal muda drasticamente o valor lógico interpretativo (semântico) dessa afirmação, pois em 2021 essa afirmação pode ser interpretada como verdadeira, porém no ano de 2019 a mesma era falsa. Assim, os valores interpretativos (semânticos) dentro do universo da lógica não são imutáveis, isto é, os valores das interpretações da lógica são passíveis de mudança a depender do contexto.

Dado então estes componentes sintáticos e semânticos pode-se concluir a partir das definições linguísticas que a \textbf{lógica é uma linguagem}, entretanto, vale salientar que não é uma linguagem natural como o português, como será visto nós próximos capítulos a lógica é uma linguagem formal \cite{benjaLivro2010}, no sentido de que todas as construções linguísticas possuem uma forma precisa e sem ambiguidade determinada por uma gramática geradora \cite{hopcroft2008, linz2006}, pode-se inclusive estabelecer que a lógica é a linguagem da ciência da inferência racional, ou seja, a linguagem usada para representar argumentos, inferência e conclusões sobre um certo universo do discurso.

\section{Um Pouco de História}\label{sec:LogicsHistory}

A história do desenvolvimento da lógica remonta até a Grécia antiga e a nomes como: Aristóteles (384-322 a.C.), Sócrates (469-399 a.C.), Zenão de Eléia (490-420 a.C.), Parmenides (515-445 a.C.), Platão (428-347 a.C.), Eudemus de Rodes (350-290 a.C.), Teofrastus de Lesbos (378-287 a.C.), Euclides de Megara (435-365 a.C.) e Eubulides de Mileto\sidefootnote{A quem é creditado o paradoxo do mentiroso.} (384-322 a.C.). De fato, o nome lógica vem do termo grego \textit{logike}, cunhado por Alexandre de Afrodisias no fim do século II depois de Cristo. Como explicado em \cite{abe2002-logica}, os mais antigos registros sobre o estudo da lógica como uma disciplina (ciência) são encontrados exatamente na obra de Aristóteles intitulado como ``$\Gamma$ da metafísica''. Todavia, após seu desenvolvimento inicial dado pelos gregos antigos, a lógica permaneceu quase que intocada\sidefootnote{Aqui não está sendo levada em conta as tentativas de Gottfried Wilhelm Leibniz (1646-1716) de desenvolver uma linguagem universal através da precisão matemática.} por mais de 1800 anos.

Os primeiros a profanar a santidade da lógica de forma contundente, abalando as estruturas da ideia de que a lógica era uma ciência completa, no sentido de que não havia nada novo a se fazer, estudar ou provar. Foram os matemáticos George Boole (1815-1864) e Augustus De Morgan (1806-1871), que introduziram a moderna ideia da lógica como uma ciência simbólica, isto é, eles semearam os conceitos iniciais que depois iriam convergir para as ideias da lógica enquanto linguagem formal apresentadas pelo matemático e filósofo alemão Gottlob Frege (1848-1925), que via a lógica como uma linguagem, que continha em seu interior todo o rigor da matemática.

Ainda no século XIX os maiores defensores das ideias de Frege, os britânicos Alfred Whitehead (1861-1947) e Betrand Russel (1872-1970), usaram muitas de suas ideias e sua linguagem na publicação monumental em três volumes intitulada ``\textit{Principia Mathematica}'' \cite{russel1910principia}, que é ainda hoje considerada por muitos o maior tratado matemático do século XIX. Como dito em \cite{benja-Logica}, outro influenciado por Frege que apresentou importantes contribuições foi filósofo austríaco Ludwig Wittgenstein (1889-1951), que em seu ``\textit{Tractatus Logico-Philosophicus}'' apresentou pela primeira vez a lógica proposicional através das tabelas verdade. Muitos autores, como é o caso de \cite{abe2002-logica}, consideram que a lógica moderna se iniciou verdadeiramente com a publicação do \textit{Principia}, de fato, alguns usam exatamente a visão de Whitehead que diz: ``A lógica atual está para a lógica aristotélica como a matemática moderna está para a aritmética das tribos primitivas''.

Outra vertente emergente na lógica do século XIX era aquela apoiada puramente por interesses matemáticos, isto é, a visão da lógica não apenas como linguagem, mas também como um objeto algebrizável (um cálculo). Tal escola de lógica encontra alguns de seus expoentes nos nomes de: Erns Zermelo\sidefootnote{Zermelo junto com Fraenkel desenvolveu o sistema formal hoje conhecido como teoria axiomática dos conjuntos.} (1871-1953), Thoralf Skolem (1887-1963), Ludwig Fraenkel-Conrad (1910-1999), John von Neumman (1903-1957), Arend Heyting (1898-1980) entre outros. Uma das grandes contribuições feitas por essa escola foi incluir uma formulação explícita e precisa das regras de inferência no desenvolvimento de sistemas axiomáticos.

Uma ramificação desta escola ``matemática'' ganhou força na Polônia sobre a tutela e liderança do lógico e filósofo Jan \L{}ukasiewics (1878-1956), o foco da escola polonesa era como dito em \cite{benja-Logica}, analisar os sistemas axiomáticos da lógica proposicional, lógica modal e das álgebras booleanas. Foi esta escola que primeiro considerou interpretações alternativas da linguagem (da lógica) e questões da meta-lógica, tais como: consistência, corretude e completude. Por fim, foi na escola polonesa que houve pela primeira vez duas visões separadas sobre a lógica, uma em que a lógica era vista puramente como uma linguagem, e a segunda visão que via a lógica puramente como um cálculo \cite{benja-Logica}.

Instigado pelo problema número dois da lista Hilbert (1862-1943), o jovem matemático e lógico austríaco Kurt Gödel (1906-1978) fez grandes contribuições para a lógica, inicialmente ele provou o teorema da completude para a lógica de primeira ordem em sua tese de doutorado em 1929, tal resultado estabelece que uma fórmula de primeira ordem é dedutível se e somente se ela é universalmente válida \cite{benja-Logica}. Outra contribuição monumental de Gödel são seus teoremas da incompletude \cite{godel1931}, em especial o primeiro que deu uma resposta negativa ao problema número dois da lista Hilbert, de forma sucinta o resultado de Gödel estabelece que não pode haver uma sistematização completa da Aritmética, ou seja, sempre vão existir sentenças verdadeiras, porém indemonstráveis \cite{abe2002-logica, magnus2020}.

Outros contemporâneos de Gödel também contribuíram fortemente para a lógica, Alfred Tarski (1901-1983) foi o responsável pela matematização do conceito de verdade como correspondência \cite{abe2002-logica, tarski1983}, já o francês Jacques Herbrand (1908-1931) introduziu as funções recursivas e apresentou os resultados hoje chamados de teoria de Herbrand. Entre os resultados de Herbrand se encontra o teorema que relaciona um conjunto insatisfatível de fórmulas da lógica de primeira ordem com um conjunto insatisfatível de fórmulas proposicionais.

Outra enorme revolução matemática do século XX que foi escrita na linguagem da lógica foi a prova da independência entre a hipótese do \textit{continuum}\sidefootnote{A hipótese do \textit{continuum} é uma conjectura proposta por Georg Cantor e que fazia parte da lista inicial de 10 problemas estabelecida por David Hilbert. Esta conjectura consiste no seguinte enunciado: \textbf{Não existe nenhum conjunto com cardinalidade maior que a do conjunto dos números inteiros e menor que a do conjunto dos números reais}.} e o axioma da escolha da teoria de conjuntos de Zermelo–Fraenkel ou teoria dos conjuntos axiomática, como também é chamada.

De forma sucinta pode-se então concluir que a lógica uma ciência nascida na Grécia antiga se desenvolveu de forma exponencial após o século XIX, e que seu desenvolvimento foi em boa parte guiado por matemáticos, de fato, pode-se dizer que a lógica contemporânea se caracteriza pela tendência da matematização da lógica \cite{barreto}. Muitos outros estudiosos, além dos que foram aqui mencionados, também apresentaram resultados diretos em lógica ou em área correlatas, como a teoria da prova e a teoria da recursão, tornando a lógica e suas ramificações e aplicações um dos assuntos dominantes nos séculos XX e XXI.

\section{Argumentos, Proposições e Predicados}\label{sec:LogicsComponents}

Como qualquer outra disciplina para entender de fato o que é a lógica deve-se estudar a mesma \cite{copi1981}, antes de qualquer coisa é bom saber que diferente de outras ciências, a lógica não apresentar fronteiras bem definidas, na verdade, como dito em \cite{joaoPavao2014}, a lógica pode ser compreendida como a tênue linha que separa as ciências da filosofia e da matemática, no que diz respeito a isto, este manuscrito irá se debruçar primariamente sobre os aspectos matemáticos da lógica.

É sabido que para se estudar uma ciência deve-se saber quais são as entidades fundamentais de interesse dessa ciência, no caso da lógica, estas entidades fundamentais são os argumentos em um discurso.

\begin{definicao}[Argumento]\label{def:Argumento}
	Um argumento é par formado por dois componentes básicos, a saber:
	\begin{itemize}
		\item[(1)] Um conjunto de frases declarativas, em que cada frase é chamada de premissa.
		\item[(2)] Uma frase declarativa, chamada de conclusão.
	\end{itemize}
\end{definicao}

Para representar um argumento pode-se como visto em \cite{copi1981, joaoPavao2014} usar uma organização de linhas, por exemplo, para representar um argumento que possua $n$ premissas primeiro serão distribuídas nas $n$ primeiras linhas as tais premissas do argumento depois na linha $n+1$ é usado o símbolo $\wasytherefore$ para separar as premissas da conclusão\sidefootnote{Como dito em \cite{magnus2020} o símbolo $\wasytherefore$ como ``portanto''.}, sendo esta última colocada na linha $n+2$.

\begin{exemplo}\label{exe:Argumento1}
	A construção:
	\begin{center}
		Toda quarta-feira é servida sopa para as crianças.\\
		Hoje é quinta-feira.\\
		$\wasytherefore$\\
		Ontem as crianças tomaram sopa.
	\end{center}
	É um argumento.
\end{exemplo}

As frases declarativas usadas para construção de argumentos são aquelas que, como dito em \cite{joaoPavao2014}, enunciam como as entidades em um certo discurso são ou poderiam ter sido, em outras palavras, as frases declarativas falam sobre as propriedades das entidades.

\begin{exemplo}\label{exe:FrasesDeclarativas}
	As frases:
	\begin{itemize}
		\item A lua é feita de queijo.
		\item O Flamengo é um time carioca.
	\end{itemize}
	São ambas frases declarativas. Por outro lado, as frases:
	\begin{itemize}
		\item Que horas são?
		\item Forneça uma resposta para o exercício.
		\item Faça exatamente o que eu mandei.
		\item Cuidado!
	\end{itemize}
	Não são frases declarativas.
\end{exemplo}

Uma forma de identificar se uma frase é declarativa é verificada se a mesma admite ser classificada como verdadeira ou falso. Na lógica, as frases declarativas podem ser ``tipadas'' com dois rótulos: \textbf{proposições} e \textbf{predicados}.

\begin{definicao}[Proposição]\label{def:Proposicao}
	Uma proposição é uma frase declarativa sobre as propriedades de indivíduos específicos em um discurso.
\end{definicao}

\begin{exemplo}\label{exe:Proposicoes}
	São exemplos de proposições:
	\begin{itemize}
		\item[(a)] $3 < 5$.
		\item[(b)] A lua é feita de queijo.
		\item[(c)] Albert Einstein era francês.
		\item[(d)] O Brasil é penta campeão de futebol masculino.
	\end{itemize}
\end{exemplo}

\begin{definicao}[Predicados]\label{def:Predicados}
	Predicados são frases declarativas sobre as propriedades de indivíduos não específicos em um discurso.
\end{definicao}

Pela Definição \ref{def:Predicados} pode-se entender que um predicado fala das propriedades de indivíduos sem explicitamente dar nomes a tais indivíduos.

\begin{exemplo}\label{exe:Predicados}
	São exemplos de predicados:
	\begin{itemize}
		\item[(a)] Para qualquer $x \in \mathbb{N}$ tem-se que $x < x + 1$.
		\item[(b)] Para todo $x \in \mathbb{R}$ sempre existem dois números $y_1, y_2 \in \mathbb{R}$ tal que $y_1 < x < y_2$.
		\item[(c)] Existe algum professor cujo nome da mãe é Maria de Fátima.
		\item[(d)] Há um estado brasileiro que não tem litoral.
	\end{itemize}
\end{exemplo}

Agora note que nas frases (a) e (b) do Exemplo \ref{exe:Predicados} o símbolo $x$ se torna um mecanismo que faz o papel dos números naturais e reais respectivamente, mas sem ser os próprios números em si, o mesmo vale para $y_1$ e $y_2$. Similarmente, na frase (c) o termo \textbf{professor} representa todo um conjunto de pessoas, mas nunca sendo uma pessoa em particular, já na frase (d) o termo \textbf{estado brasileiro} representa novamente todos os indivíduos de um conjunto, mas ele nunca é um indivíduo particular. Os termos em um predicado que tem essa capacidade de representação são chamados de \textbf{variáveis do predicado}.

\begin{atencao}
	Um predicado que tem suas variáveis substituídas (ou instanciadas) por valores específicos ou concretos se torna uma proposição.
\end{atencao}




