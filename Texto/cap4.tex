\chapter{Demonstrações}\label{cap:Proofs}

\epigraph{``Mais um colchão, mais uma demonstração''.}{Paul Erdös}

\epigraph{``Um matemático é uma máquina que transforma café em teoremas''.}{Paul Erdös}

\section{Introdução}\label{sec:Introducao}

Desde que, as demonstrações são figuras de interesse central no cotidiano dos matemáticos, cientistas da computação e engenheiros de software, em especial aqueles que trabalham com métodos formais, este texto irá fazer uma breve pausa no estudo da teoria dos conjuntos, para apresentar um pouco de teoria da prova ao leitor.

Este capítulo começa então com o seguinte questionamento: ``Do ponto de vista da ciência da computação, qual a importância das demonstrações?'' Bem, a resposta a essa pergunta pode ser dada de dois pontos de vista, um teórico (purista) e um prático (aplicado ou de engenharia).

Na perspectiva de um cientista da computação puro, as demonstrações de teoremas, proposições, lema, corolários e propriedades são a principal ferramenta para investigar os limites dos diferentes modelos de computação propostos \cite{hopcroft2008, linz2006}, assim sendo é de suma importância que o estudante de graduação em ciência da computação receba em sua formação pelo menos o básico para dominar a ``arte'' de provar teoremas, sendo assim preparado para o estudo e a pesquisa pura em computação e(ou) matemática.

Já na visão prática, só existe uma forma segura de garantir que um \textit{software} está livre de erros, essa ``tecnologia'' é exatamente a demonstração das propriedades do \textit{software}. 

É claro que, mostrar que um \textit{software} não possui erros exige que o \textit{software} seja visto através de um certo nível de formalismo e rigor matemático, mas após essa modelagem, demonstrações podem garantir que um \textit{software} não apresentará erros (quando bem especificado), e assim se algo errado ocorrer foi por fatores externos, tais como defeito no \textit{hardware} por exemplo, e não por falha ou erros com a implementação. Este conceito é o cerne de uma área da engenharia de \textit{software} \cite{pressman2016}, chamada métodos  (ou especificações) formais, sendo essa área o ponto crucial no desenvolvimento de \textit{softwares} para sistemas críticos \cite{sommerville2011}. Isto já mostra a grande importância de programadores e engenheiros de \textit{software} terem em sua formação as bases para o domínio das técnicas de demonstração.

Nas próxima seções deste documento serão descritas as principais técnicas de demonstração de interesse de matemáticos, cientistas da computação e engenheiros formais de \textit{software}.

\begin{atencao}
  Para o leitor que nunca antes teve contato com a lógica matemática, recomenda-se que antes de estudar este capítulo, o leitor deve fazer pelo menos uma leitura superficial em obras como \cite{edgar2002, leonidas2002, joaoPavao2014}. 
\end{atencao}

Para poder falar sobre métodos de demonstração e poder então descrever como os matemáticos, lógicos e cientistas da computação justificam propriedades usando apenas a argumentação matemática, será necessário fixar algumas nomenclaturas e falar sobre alguns conceitos importantes.

\begin{definicao}[Asserção]\label{def:Assercao}
	Uma \textbf{asserção} é qualquer frase declarativa que possa ser expressa na linguagem da lógica simbólica.
\end{definicao}

Os métodos (ou estratégias) de demonstrações apresentadas neste documento seguem as ideias e a ordem  de apresentação similar ao que foi exposto em \cite{velleman2019comProvar}. Em \cite{velleman2019comProvar} antes de apresentar as provas formais, erá necessário a construção de um rascunho de prova, este rascunho possui similaridades com as demonstrações em provadores de teoremas tais como Coq \cite{coq2013, softwarefoundations} e Lean \cite{lean2015}, isto é, existe uma separação clara entre dados (hipótese) e os objetivos (em inglês \textit{Goal}) que se quer demonstrar.

Neste documento por outro lado, não será utilizado a ideia de um rascunho de prova, em vez disso, será usado aqui a noção de \textbf{diagrama de blocos} \cite{broda2007}. Aqui tais diagramas serão encarados como as demonstrações em si, assim diferente de \cite{velleman2019comProvar} não haverá a necessidade de escrever um texto formal após o diagrama da prova ser completado.

Sobre o diagrama de blocos é conveniente explicar sua estrutura, ele consiste de uma série de linhas numeradas de $1$ até $m$, em cada linha está uma informação, sendo esta uma hipótese assumida como verdadeira ou deduzida a partir das informações anteriores a ela ou ainda um resultado (ou definição) válido(a) conhecido(a). Um diagrama de bloco representa uma prova, porém, uma prova pode conter $n$ subprovas. Cada \textbf{prova} é delimitada no diagrama por um \textbf{bloco}, assim se existe uma subprova $p'$ em uma prova $p$, significa que o diagrama de bloco de $p'$ é interno ao diagrama de bloco de $p$. Na linha abaixo de todo bloco sempre estará a conclusão que se queria demonstrar, isto é, abaixo de cada bloco está a asserção que tal bloco demonstra.

Cada linha no diagrama começa com algum termo reservado (em um sentido similar ao de palavra reservada de linguagem de programação \cite{aho2007, cooper2017}) escrito em negrito\sidefootnote{A escrita dos termos reservados em negrito em geral será usada para que o leitor consiga identificar o que é informação último da prova e o que é apenas um artificio textual para dá melhor entendimento a demonstração.}, esses termos reservados tem três naturezas distintas: inicialização de bloco, ligação e conclusão de blocos. Tais palavras podem variar a depender do material sobre demonstrações que o leitor possa encontrar na literatura neste documento serão usando os seguintes conjuntos de palavras:

\begin{itemize}
	\item Termos de inicialização de bloco: \textbf{Suponha}, \textbf{Deixe ser}, \textbf{Assuma} e \textbf{Considere};
	\item Termos de ligação: \textbf{mas}, \textbf{tem-se que}, \textbf{então}, \textbf{assim}, \textbf{logo}, \textbf{além disso}, \textbf{desde que} e \textbf{dessa forma};
  \item Termos de conclusão de bloco: \textbf{Portanto}, \textbf{Dessa forma}, \textbf{Consequentemente}, \textbf{Por contrapositiva} e \textbf{Logo por contrapositiva}.
\end{itemize}

\begin{exemplo}\label{exe:DiagramaProva1}
  Demonstre a asserção: Dado $A = \{x \in \mathbb{Z} \mid x = 2i, i \in \mathbb{Z}\}$ e $B = \{x \in \mathbb{Z} \mid x = 2j - 2, j \in \mathbb{Z}\}$ tem-se que $A = B$.
  {\scriptsize
	\begin{logicproof}{3}
		\begin{subproof}
			\text{\textbf{Assuma} que } x \text{ é um elemento qualquer, }&\\
			\begin{subproof}
				\text{\textbf{Suponha} que } x \in A, &\\
				\text{\textbf{assim }} x = 2i, i \in \mathbb{Z}, &\\
				\text{\textbf{desde que} existe } k \in \mathbb{Z} \text{ tal que } k = j - 1 \text{ e fazendo } i = k, &\\
				\text{\textbf{tem-se que} } x = 2i = 2k = 2(j -1) = 2j - 2.&\\
				\text{\textbf{Então} } x \in B. &
			\end{subproof}
			\text{\textbf{Portanto}, Se } x \in A, \text{ então } x \in B.&
		\end{subproof}
		\text{\textbf{Consequentemente}, } \forall x.[\text{ Se } x \in A, \text{ então } x \in B].&\\
		\begin{subproof}
			\text{\textbf{Assuma} que } x \text{ é um elemento qualquer, }&\\
			\begin{subproof}
				\text{\textbf{Suponha} que } x \in B. &\\
				\text{\textbf{assim }} x = 2j -2, j \in \mathbb{Z}, &\\
				\text{\textbf{desde que} } j - 1 \in \mathbb{Z}, \text{ pode-se fazer } i = j - 1&\\
				\text{\textbf{logo }} x = 2i. &\\
				\text{\textbf{Então} } x \in A. &
			\end{subproof}
			\text{\textbf{Portanto}, Se } x \in B, \text{ então } x \in A.&
		\end{subproof}
		\text{\textbf{Consequentemente}, } \forall x.[\text{ Se } x \in B, \text{ então } x \in A].&\\
		\text{\textbf{Portanto,} } A \subseteq B \text{ e } B \subseteq A \text{ assim por definição } A = B.&
	\end{logicproof}
	}
\end{exemplo}

Neste momento o exemplo anterior serve apenas para esboçar a ideia de um diagrama de bloco para uma demonstração, note que fica evidente que a depender da situação alguns termos de ligação são melhores que outros, e o mesmo também é válido para os termos de inicialização e conclusão de bloco.

Aqui não será detalhando a aplicação dos métodos de demonstração usado na demonstração do Exemplo \ref{exe:DiagramaProva1}, mas nas próximas seções serão apresentados cada um dos métodos de demonstração, e seguida será gradativamente apresentados exemplos para esboçar ao leitor como é usado o diagrama de blocos e relação a cada método de demonstração.

Como o leitor pode ter notado pelo diagrama de bloco no Exemplo \ref{exe:DiagramaProva1}, é possível enxergar o diagrama como ambiente muito similar a ideia de um programa imperativo em uma linguagem de programação estruturada (como Pascal ou C), no sentido de que, uma demonstração pode ser visto como a combinação de diversos blocos, em que os blocos respeito uma hierarquia e podem está aninhados entre si, a hierarquia dos bloco é determinar por uma indentação\sidefootnote{Indentação é um termo utilizando em código fonte de um programa, serve para ressaltar, identificar ou definir a estrutura do algoritmo.}.

\section{Demonstrando Implicações}\label{sec:DemonstrandoImplicacoes}

Este documento irá iniciar a apresentação dos métodos de demonstração a partir das estratégias usadas para demonstrar a implicação, isto é, as estratégias usadas para provar asserção da forma: ``se $\alpha$, então $\beta$''.

\begin{definicao}[Prova Direta (PD)]
	Dado uma asserção da forma: ``se $\alpha$, então $\beta$''. A metodologia de prova direta para tal asserção consiste em supor $\alpha$ como sendo verdade e a partir disto deduzir $\beta$.
\end{definicao}

Esta estratégia é provavelmente a técnicas mais famosa e usada dentre todos os métodos de demonstração que existem, um conhecedor de lógica pode notar facilmente que tal estratégia nada mais é do que a regra de dedução natural chamada de introdução da implicação\cite{joaoPavao2014}.

No que diz respeito ao diagrama tal estratégia consistem em: (1) criar um bloco, e dentro deste bloco na primeira linha irá conter a afirmação de que $\alpha$ está sendo assumido com uma hipótese verdadeira; (2) nas próximas $n$ linhas irão acontecer as deduções necessárias até que na linha $n+2$ seja deduzido o $\beta$ e o bloco é fechado e (3) na linha $n + 3$ será adicionada a conclusão do bloco. A seguir serão apresentados exemplos do uso do método de demonstração direto para implicações e seu uso junto com o diagrama de bloco.

\begin{exemplo}\label{exe:DiagramaProva2}
	Será Provado a asserção, \textbf{``Se $x$ é ímpar, então $x^2 + x$ é par''}, usando \textbf{PD}. A prova começa abrindo um bloco e inserido na primeira linha a hipótese de que o antecedente \textbf{$x$ é um número ímpar} é verdadeira, ou seja, tem-se:
  {\scriptsize
	\begin{logicproof}{2}
		\begin{subproof}
			\text{\textbf{Suponha} que } x \text{ é um número ímpar} &
		\end{subproof}
	\end{logicproof}
	}
	\noindent em seguida  pode-se na linha 2 deduzir a forma de $x$, mudando o diagrama para:
  {\scriptsize
		\begin{logicproof}{2}
			\begin{subproof}
				\text{\textbf{Suponha} que } x \text{ é um número ímpar},  &\\
				\text{\textbf{logo} } x = 2k + 1, k  \in \mathbb{Z},&
			\end{subproof}
		\end{logicproof}
	}
	\noindent  agora nas próximas duas linhas pode-se deduzir respectivamente as formas (ou valores) de $x^2$ e $x^2 + x$, assim o diagrama é atualizado para:
	{\scriptsize
		\begin{logicproof}{2}
			\begin{subproof}
				\text{\textbf{Suponha} que } x \text{ é um número ímpar}, &\\
				\text{\textbf{logo} } x = 2k + 1, k  \in \mathbb{Z},&\\
				\text{\textbf{assim} } x^2 = 4k^2 + 4k + 1, k \in \mathbb{Z},&\\
				\text{\textbf{dessa forma} } x^2 + x= 2((2k^2 + 2k) + k + 1), k \in \mathbb{Z}.&
			\end{subproof}
		\end{logicproof}
	}
  \noindent note que $x^2 + x= 2((2k^2 + 2k) + k + 1)$ pode ser reescrito (por substituição) como $x^2 + x= 2j$ com $j = (2k^2 + 2k) + k + 1$, essa dedução é inserida na linha de número $5$ atualizando o diagrama para:
	{\scriptsize
		\begin{logicproof}{2}
			\begin{subproof}
				\text{\textbf{Suponha} que } x \text{ é um número ímpar}, &\\
				\text{\textbf{logo} } x = 2k + 1, k  \in \mathbb{Z},&\\
				\text{\textbf{assim} } x^2 = 4k^2 + 4k + 1, k \in \mathbb{Z},&\\
				\text{\textbf{dessa forma} } x^2 + x= 2((2k^2 + 2k) + k + 1), k \in \mathbb{Z}.&\\
				\text{\textbf{logo} } x^2 + x= 2j \text{ com } j = (2k^2 + 2k) + k + 1, k \in \mathbb{Z}.&
			\end{subproof}
		\end{logicproof}
	}
	\noindent assim pode-se então deduzir a partir da informação na linha de número $5$ que $x^2 + x$ é um número par, assim o diagrama muda para a forma:
	{\scriptsize
		\begin{logicproof}{2}
			\begin{subproof}
				\text{\textbf{Suponha} que } x \text{ é um número ímpar}, &\\
				\text{\textbf{logo} } x = 2k + 1, k  \in \mathbb{Z},&\\
				\text{\textbf{assim} } x^2 = 4k^2 + 4k + 1, k \in \mathbb{Z},&\\
				\text{\textbf{dessa forma} } x^2 + x= 2((2k^2 + 2k) + k + 1), k \in \mathbb{Z}.&\\
				\text{\textbf{logo} } x^2 + x = 2j \text{ com } j = (2k^2 + 2k) + k + 1, k \in \mathbb{Z},&\\
					\text{\textbf{então} } x^2 + x \text{ por definição é um número par}.&
			\end{subproof}
		\end{logicproof}
	}
  \noindent note porém que a informação na deduzida na linha de número $6$ é exatamente o consequente da implicação que se queria deduzir. Portanto, o objetivo interno ao bloco foi atingido, pode-se então fechar o bloco introduzindo abaixo dele a conclusão do bloco, ou seja, na linha de número $7$ é escrito que o antecedente de fato implica no consequente, assim o diagrama fica da forma:
	{\scriptsize
		\begin{logicproof}{2}
			\begin{subproof}
				\text{\textbf{Suponha} que } x \text{ é um número ímpar}, &\\
				\text{\textbf{logo} } x = 2k + 1, k  \in \mathbb{Z},&\\
				\text{\textbf{assim} } x^2 = 4k^2 + 4k + 1, k \in \mathbb{Z},&\\
				\text{\textbf{dessa forma} } x^2 + x= 2((2k^2 + 2k) + k + 1), k \in \mathbb{Z}.&\\
				\text{\textbf{logo} } x^2 + x = 2j \text{ com } j = (2k^2 + 2k) + k + 1, k \in \mathbb{Z},&\\
				\text{\textbf{então} } x^2 + x \text{ por definição é um número par}.&
			\end{subproof}
			\text{\textbf{Portanto}, Se } x \text{ é ímpar, então } x^2 + x \text{ é par.} &
		\end{logicproof}
	}
	\noindent assim o objetivo a ser demonstrado foi atingido e, portanto, a prova está completa.
\end{exemplo}

Na demonstração apresentada no Exemplo \ref{exe:DiagramaProva2} as justificativas da evolução do diagrama fora apresentadas passo a passo e separadas do diagrama, isso foi adotado nesse primeiro exemplo para detalhar a evolução da demonstração ao leitor, entretanto, isso não é o padrão, o normal (que será adotado) é que a justificativa (caso necessário\sidefootnote{Quando a justificativa for trivial, não é necessário.}) da dedução de uma linha seja inserida a direita da informação deduzida, destacada em {\color{blue}azul}. Além disso, nas justificativas das provas as palavras definição, associatividade, comutatividade serão  abreviadas para DEF, ASS, COM respetivamente.

\begin{exemplo}\label{exe:DiagramaProva3}
	Demonstração da asserção: Se $x = 4k$, então $x$ é múltiplo de 2.
	
	{\scriptsize
		\begin{logicproof}{2}
			\begin{subproof}
        \text{\textbf{Suponha} que } x = 4k, k  \in \mathbb{Z},& {\color{blue}Hipótese}\\
        \text{\textbf{assim} } x = (2 \cdot 2)k, k \in \mathbb{Z},& {\color{blue}Reescrita da linha $1$}\\
        \text{\textbf{dessa forma} } x= 2(2k), k \in \mathbb{Z}.& {\color{blue}ASS da multiplicação}\\
        \text{\textbf{logo} } x= 2i, \text{ com } i = 2k, k \in \mathbb{Z}.& {\color{blue}Reescrita da linha $3$}\\
        \text{\textbf{então} } x \text{ é múltiplo de } 2.& {\color{blue}DEF de múltiplo de 2}
			\end{subproof}
      \text{\textbf{Portanto}, Se } x = 4k, \text{ então $x$ é múltiplo de 2}. & {\color{blue}PD (linhas 1-6)}
		\end{logicproof}
	}
\end{exemplo}

O leitor deve ter notado nos Exemplos \ref{exe:DiagramaProva2} e \ref{exe:DiagramaProva3} as demonstrações sempre iniciam das hipótese que estão sendo assumidas, isto é, os antecedentes das implicações, isso ocorrer por que nenhuma informação adicional (necessária) é apresentada como premissa, há caso entretanto, que as premissas são importantes para o desenvolvimento da prova, como será visto no próximo exemplo. 

\begin{exemplo}\label{exe:DiagramaProva4}
  Demonstração da asserção: Dado $m$ inteiro maior ou igual que $5$ e $n$ um número impar maior que $0$. Se $m = 2i + 1$, então $m + n \geq 6$.
	
	{\scriptsize
		\begin{logicproof}{2}
      m \geq 5, m \in \mathbb{Z} & {\color{teal}Premissa}\\
      n = 2j + 1, j \in  \mathbb{Z}^+ & {\color{teal}Premissa}\\
			\begin{subproof}
        \text{\textbf{Suponha} que } m = 2i +1,& {\color{blue}Hipótese}\\
				\text{\textbf{assim} } m + n = 2(i + j + 1)&\\
				\text{\textbf{desde que} } m \geq 5 \text{ tem-se que } i \geq 2, &\\
				\text{\textbf{mas} como } n \in \mathbb{Z}_+ \text{ tem-se que } j \geq 0, &\\
        \text{\textbf{assim }} i + j + 1 \geq 3, & {\color{blue}Direto das linhas $5$ e $6$}\\
				\text{\textbf{logo} } 2(i + j + 1) \geq 6 &\\
        \text{\textbf{então} } m + n \geq 6. & {\color{blue}Reescrita da linha 8} 
			\end{subproof}
      \text{\textbf{Portanto}, Se } m = 2i + 1, \text{ então } m + n \geq 6. & {\color{blue}PD (linhas 3-9)}
		\end{logicproof}
	}
\end{exemplo}

\begin{exemplo}\label{exe:DiagramaProva5}
	Demonstração da asserção: Dado $m,n \in \mathbb{R}$ e $3m + 2n \leq 5$. Se $m > 1$, então $n < 1$.
	
	{\scriptsize
		\begin{logicproof}{2}
      m, n \in \mathbb{R} & {\color{teal}Premissa}\\
      3m + 2n \leq 5 & {\color{teal}Premissa}\\
			\begin{subproof}
        \text{\textbf{Suponha} que } m > 1, & {\color{blue}Hipótese}\\
				\text{\textbf{assim} } 3m > 3, &\\
				\text{\textbf{desde que }} 3m \leq 5 - 2n, &\\
        \text{\textbf{tem-se que }} 3 < 3m  \leq  5 - 2n& {\color{blue}Direto das linhas $4$ e $5$}\\
				\text{\textbf{logo} } 3 < 5 - 2n,&\\
				\text{\textbf{assim} } 3 + 2n < 5,&\\
				\text{\textbf{então} } n < 1.&
			\end{subproof}
      \text{\textbf{Portanto}, Se } n > 1, \text{ então }  n < 1. & {\color{blue}PD (linhas 3-10)}
		\end{logicproof}
	}
\end{exemplo}

Além do método de prova direta asserções que são implicações podem ser provadas por um segundo método, chamado método da contrapositiva (ou contraposição). Como dito em \cite{menezes2010MD}, o método da contrapositiva se baseia na equivalência semântica\sidefootnote{Para um entendimento sobre equivalência semântica ver \cite{edgar2002, leonidas2002}.} da expressão ``Se $\alpha$, então $\beta$'' com a expressão ``Se não $\beta$, então não $\alpha$''. Formalmente o método de demonstração por contrapositiva é como se segue.

\begin{definicao}[Prova por contrapositiva (PCP)]
	Dado uma asserção da forma: ``se $\alpha$, então $\beta$''. A metodologia de prova por contrapositiva para tal asserção consiste em demonstrar usando PD a asserção ``se não $\beta$, então não $\alpha$'', em seguida concluir (ou enunciar) que a veracidade de ``se $\alpha$, então $\beta$'' segue da veracidade de ``se não $\beta$, então não $\alpha$''.
\end{definicao}

Agora em termos do diagrama de blocos o método PCP apresenta o seguinte raciocínio de construção do diagrama: (1) abrir um bloco com a primeira linha em branco; (2) realizar em um bloco (interno) a  demonstração  de que ``se não $\beta$, então não $\alpha$'' e (3) após a conclusão deste segundo bloco, o primeiro bloco é fechado, e sua conclusão consiste na informação ``se $\alpha$, então $\beta$'' e a justificativa de tal informação é simplesmente a conclusão PCP das linhas $i$-$j$, onde $i$-$j$ diz respeito ao intervalo contendo as  linhas do bloco e da conclusão da prova de ``se não $\beta$, então não $\alpha$''.

\begin{atencao}
  Vale salientar que a linha em branco no início dos próximos exemplos é apenas um \textbf{fator estético} adotado, para tornar a leitura do diagrama da demonstração mais agradável. Esse recurso pode voltar a ser usado em exemplos futuros. É mais conveniente escrever, por exemplo apenas $l2$, do que ter que escrever linhas $2$ nas justificativas, assim o $l$ nas justificativas a partir deste ponto deve ser lido como ``linha'' ou ``linhas'' no caso de ser $lx-y$ onde $x$ e $y$ são os números.
\end{atencao}

\begin{exemplo}\label{exe:DiagramaProva6}
	Demonstração da asserção: Se $n! > (n+1)$, então $n > 2$.
	{\scriptsize
		\begin{logicproof}{3}
			\begin{subproof}
				&  \\
				\begin{subproof}
          \text{\textbf{Suponha} que } n \leq 2,&{\color{blue}Hipótese}\\
          \text{\textbf{assim} } n = 0, n = 1 \text{ ou } n = 2&{\color{blue}Direto da $l2$}\\
          \text{\textbf{logo} } n! = 1 \text{ ou } n! = 2,&{\color{blue}Da $l3$ e da DEF de fatorial}\\
          \text{\textbf{então} } n! \leq (n + 1) \text{ com } n \leq 2 .&{\color{blue}Direto de $l3$ e $l4$}
				\end{subproof}
        \text{\textbf{Portanto}, Se } n \leq 2, \text{ então }  n! \leq (n + 1).&{\color{blue}PD ($l2-5$)}
			\end{subproof}
      \text{\textbf{Por contrapositiva}, Se }n! > (n+1), \text{ então }n > 2. &{\color{blue}PCP ($l2-6$)}
		\end{logicproof}
	}
\end{exemplo}

\begin{exemplo}\label{exe:DiagramaProva7}
	Demonstração da asserção: Se $n \neq 0$, então $n + c \neq c$.
	{\scriptsize
		\begin{logicproof}{3}
			\begin{subproof}
				&  \\
				\begin{subproof}
          \text{\textbf{Suponha} que } n + c = c, &{\color{blue}Hipótese}\\
					\text{\textbf{assim} } n + c - c = c -c  & \\
					\text{\textbf{logo}, } n + 0 = 0, &\\
					\text{\textbf{então} } n = 0 .&
				\end{subproof}
        \text{\textbf{Portanto}, Se } n + c = c, \text{ então }  n  = 0. &{\color{blue}PD ($l2-5$)}
			\end{subproof}
      \text{\textbf{Logo por contrapositiva}, Se }n \neq 0, \text{ então } n + c \neq c. &{\color{blue}PCP ($l2-6$)}
		\end{logicproof}
	}
\end{exemplo}

\begin{exemplo}\label{exe:DiagramaProva8}
	Demonstração da asserção: Dado três números $x, y, z \in \mathbb{R}$ com $x > y$. Se $xz \leq yz$, então $z \leq 0$.
	{\scriptsize
		\begin{logicproof}{3}
      x, y, z \in \mathbb{R},&{\color{teal}Premissa}\\
      x > y, &{\color{teal}Premissa}\\
			\begin{subproof}
				&  \\
				\begin{subproof}
          \text{\textbf{Suponha} que } z > 0, &{\color{blue}Hipótese}\\
          \text{\textbf{então} } xz > yz,  &{\color{blue}Das $l2$ e $l4$ e da MON\footnote{MON aqui é a abreviatura de monotonicidade.} da $\cdot$ em $\mathbb{R}$}
				\end{subproof}
        \text{\textbf{Portanto}, Se } z > 0, \text{ então }  xz > yz. &{\color{blue}PD ($l2$-$5$)}
			\end{subproof}
      \text{\textbf{Por contrapositiva}, Se }xz \leq yz, \text{ então } z \leq 0. &{\color{blue}PCP ($l3-6$)}
		\end{logicproof}
	}
\end{exemplo}

\begin{exemplo}\label{exe:DiagramaProva9}
	Demonstração da asserção: Se $n^2$ é par, então $n$ é par.
	{\scriptsize
		\begin{logicproof}{3}
			\begin{subproof}
				&  \\
				\begin{subproof}
          \text{\textbf{Suponha} que } n \text{ não é par}, &{\color{blue}Hipótese}\\
          \text{\textbf{logo} } n = 2k + 1 \text{ com } k \in \mathbb{Z},  &{\color{blue}DEF de paridade}\\
					\text{\textbf{assim} } n^2 = 4k^2 + 4k + 1 \text{ com } k \in \mathbb{Z},&\\
          \text{\textbf{dessa forma }} n^2 = 2j + 1 \text{ com } j = 2k^2 + 2k,&{\color{blue}Reescrita de $l4$}\\
          \text{\textbf{então} } n^2 \text{ não é par}&{\color{blue}DEF de paridade}
				\end{subproof}
        \text{\textbf{Portanto}, Se } n  \text{ não é par}, \text{ então } n^2  \text{ não é par}. &{\color{blue}PD ($l2$-$6$)}
			\end{subproof}
      \text{\textbf{Logo por contrapositiva}, Se }n^2 \text{ é par}, \text{ então } n \text{ é par}. &{\color{blue}PCP ($l2$-$6$)}
		\end{logicproof}
	}
\end{exemplo}

\section{Demonstração por Absurdo}\label{sec:DemonstracaoAbsurdo}